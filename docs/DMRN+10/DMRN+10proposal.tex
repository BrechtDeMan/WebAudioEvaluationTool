
\documentclass[a4paper]{article}


%\setlength{\droptitle}{-5em}   % Move title position up


\title{Web Audio Evaluation Tool: A browser-based listening test framework}
\author{Nicholas Jillings, David Moffat, Brecht De Man and Joshua D. Reiss\\ \\
Preference: Poster }

\begin{document}

\maketitle
%\begin{center}
%{\large
%Preference: Poster}
%\end{center}
%
%Title: Web Audio Evaluation Tool: A browser-based listening test framework
%Authors: Nicholas Jillings, David Moffat, Brecht De Man and Joshua D. Reiss

\section*{Abstract}
Perceptual evaluation of audio is a popular and powerful method in research topics ranging from signal source separation over audio codec quality to emotion in music. Most researchers in the field of music and audio have conducted a listening test at one point or another. As a consequence, many different tools have been developed for various platforms and use cases. However, few tools have support a wide range of standard (or custom) interfaces, and almost none are compatible with all conventional operating systems, or require tedious set up or external applications or libraries when they are. Another frequent obstacle is the laborious setup through configuration files or even the requirement of a programming background to develop a suitable listening test interface. 

The recent introduction of the Web Audio API enables a wide range of functionalities that were previously not possible in the browser. This allows for audio applications to be compatible with various devices so long as the application supports the available web browser. 

A framework to develop and conduct listening tests in the browser, both off- and online will be presented. Most significantly, this browser-based tool is cross-platform, not dependent on any proprietary software and can be hosted on a web server so that remote tests are possible. 
Many standard interfaces are already included, and custom interfaces can easily be created using the available interface elements. 

Moreover, by allowing creation and modification of the configuration files from within the browser, no programming knowledge is required from the user, making the tool attractive to a much wider range of researchers. Diagnostics and analysis tools in the browser enable quick troubleshooting, quick assessment of the reliability of subjects, and basic analysis of the results. 

A demo of the Web Audio Evaluation Tool, including its easy test creator tool and post-test diagnostics and analysis, and discuss planned improvements and extensions will be presented.
\end{document}
