% -----------------------------------------------
% Template for SMC 2012
% adapted from the template for SMC 2011, which was adapted from that of SMC 2010
% -----------------------------------------------

\documentclass{article}
\usepackage{smc2015}
\usepackage{times}
\usepackage{ifpdf}
\usepackage[english]{babel}
\usepackage{cite}

%%%%%%%%%%%%%%%%%%%%%%%% Some useful packages %%%%%%%%%%%%%%%%%%%%%%%%%%%%%%%
%%%%%%%%%%%%%%%%%%%%%%%% See related documentation %%%%%%%%%%%%%%%%%%%%%%%%%%
%\usepackage{amsmath} % popular packages from Am. Math. Soc. Please use the 
%\usepackage{amssymb} % related math environments (split, subequation, cases,
%\usepackage{amsfonts}% multline, etc.)
%\usepackage{bm}      % Bold Math package, defines the command \bf{}
%\usepackage{paralist}% extended list environments
%%subfig.sty is the modern replacement for subfigure.sty. However, subfig.sty 
%%requires and automatically loads caption.sty which overrides class handling 
%%of captions. To prevent this problem, preload caption.sty with caption=false 
%\usepackage[caption=false]{caption}
%\usepackage[font=footnotesize]{subfig}


%user defined variables
\def\papertitle{APE FOR WEB: A BROWSER-BASED EVALUATION TOOL FOR AUDIO}
\def\firstauthor{Brecht De Man}
\def\secondauthor{Nicholas Jillings}
\def\thirdauthor{David Moffat}
\def\fourthauthor{Joshua D. Reiss}

% adds the automatic
% Saves a lot of ouptut space in PDF... after conversion with the distiller
% Delete if you cannot get PS fonts working on your system.

% pdf-tex settings: detect automatically if run by latex or pdflatex
\newif\ifpdf
\ifx\pdfoutput\relax
\else
   \ifcase\pdfoutput
      \pdffalse
   \else
      \pdftrue
\fi

\ifpdf % compiling with pdflatex
  \usepackage[pdftex,
    pdftitle={\papertitle},
    pdfauthor={\firstauthor, \secondauthor, \thirdauthor},
    bookmarksnumbered, % use section numbers with bookmarks
    pdfstartview=XYZ % start with zoom=100% instead of full screen; 
                     % especially useful if working with a big screen :-)
   ]{hyperref}
  %\pdfcompresslevel=9

  \usepackage[pdftex]{graphicx}
  % declare the path(s) where your graphic files are and their extensions so 
  %you won't have to specify these with every instance of \includegraphics
  \graphicspath{{./figures/}}
  \DeclareGraphicsExtensions{.pdf,.jpeg,.png}

  \usepackage[figure,table]{hypcap}

\else % compiling with latex
  \usepackage[dvips,
    bookmarksnumbered, % use section numbers with bookmarks
    pdfstartview=XYZ % start with zoom=100% instead of full screen
  ]{hyperref}  % hyperrefs are active in the pdf file after conversion

  \usepackage[dvips]{epsfig,graphicx}
  % declare the path(s) where your graphic files are and their extensions so 
  %you won't have to specify these with every instance of \includegraphics
  \graphicspath{{./figures/}}
  \DeclareGraphicsExtensions{.eps}

  \usepackage[figure,table]{hypcap}
\fi

%setup the hyperref package - make the links black without a surrounding frame
\hypersetup{
    colorlinks,%
    citecolor=black,%
    filecolor=black,%
    linkcolor=black,%
    urlcolor=black
}


% Title.
% ------
\title{\papertitle}

% Authors
% Please note that submissions are NOT anonymous, therefore 
% authors' names have to be VISIBLE in your manuscript. 
%
% Single address
% To use with only one author or several with the same address
% ---------------
%\oneauthor
%   {\firstauthor} {Affiliation1 \\ %
%     {\tt \href{mailto:author1@smcnetwork.org}{author1@smcnetwork.org}}}

%Two addresses
%--------------
% \twoauthors
%   {\firstauthor} {Affiliation1 \\ %
%     {\tt \href{mailto:author1@smcnetwork.org}{author1@smcnetwork.org}}}
%   {\secondauthor} {Affiliation2 \\ %
%     {\tt \href{mailto:author2@smcnetwork.org}{author2@smcnetwork.org}}}

% Three addresses
% --------------
 \fourauthors
   {\firstauthor} {%Affiliation1 \\
     {\tt \href{mailto:b.deman@qmul.ac.uk}{\{b.deman@, }}}
   {\secondauthor} {%Affiliation2\\ %
     {\tt \href{mailto:n.g.r.jillings@se14.qmul.ac.uk}{n.g.r.jillings@se14. ,}}}
   {\thirdauthor} {%Affiliation3\\ %
     {\tt \href{mailto:d.j.moffat@qmul.ac.uk}{d.j.moffat@, }}}
    {\fourthauthor} {%Affiliation4\\ %
     {\tt \href{mailto:joshua.reiss@qmul.ac.uk}{joshua.reiss@\}qmul.ac.uk}}}

% ***************************************** the document starts here ***************
\begin{document}
%
\capstartfalse
\maketitle
\capstarttrue
%
\begin{abstract}
Place your abstract at the top left column on the first page.
Please write about 150-200 words that specifically highlight the purpose of your work,
its context, and provide a brief synopsis of your results.
Avoid equations in this part.
\end{abstract}
%

\section{Introduction}\label{sec:introduction}

background (types of research)\\
prior work: \cite{deman2014b} \\
goal, what are we trying to do? \\


Minimum 4 pages, 6 preferred, max. 8 (6 for demos/posters)




%\subsection{Equations}
%Equations of importance, 
%or to which you refer later,
%should be placed on separated lines and numbered.
%The number should be on the right side, in parentheses.
%\begin{equation}
%E=mc^{2+\delta}.
%\label{eq:Emc2}
%\end{equation}
%Refer to equations like so:
%As (\ref{eq:Emc2}) shows, 
%I do not completely trust Special Relativity.
%
%\subsection{Figures, Tables and Captions}
%\begin{table}[t]
% \begin{center}
% \begin{tabular}{|l|l|}
%  \hline
%  String value & Numeric value \\
%  \hline
%  Hej SMC  & 2015 \\
%  \hline
% \end{tabular}
%\end{center}
% \caption{Table captions should be placed below the table, exactly like this,
% but using words different from these.}
% \label{tab:example}
%\end{table}

%\begin{figure}[t]
%\figbox{
%\subfloat[][]{\includegraphics[width=60mm]{figure}\label{fig:subfigex_a}}\\
%\subfloat[][]{\includegraphics[width=80mm]{figure}\label{fig:subfigex_b}}
%}
%\caption{Here's an example using the subfig package.\label{fig:subfigex} }
%\end{figure}


\section{Conclusions}\label{sec:conclusions}


%\begin{acknowledgments}
%You may acknowledge people, projects, 
%funding agencies, etc. 
%which can be included after the second-level heading
%``Acknowledgments'' (with no numbering).
%\end{acknowledgments} 

%%%%%%%%%%%%%%%%%%%%%%%%%%%%%%%%%%%%%%%%%%%%%%%%%%%%%%%%%%%%%%%%%%%%%%%%%%%%%
%bibliography here
%\bibliography{smc2015template}

\bibliography{../General}{}


\end{document}
